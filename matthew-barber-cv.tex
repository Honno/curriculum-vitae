% !TeX program = xelatex
\title{Matthew Barber}
\author{}
\date{}

\documentclass[11pt]{article}

% prevent hyphen breaks
\tolerance=1
\emergencystretch=\maxdimen
\hyphenpenalty=10000
\hbadness=10000

\usepackage[hidelinks]{hyperref}

\usepackage{graphicx}

\usepackage{geometry}
\geometry{
	a4paper,
	total={120mm,257mm},
	left=20mm,
	top=20mm,
	% 50mm to play
	marginparsep=10mm,
	marginparwidth=50mm % overflow
}

\setlength{\parindent}{0in}

\usepackage{color}
\usepackage{xcolor}

% Material Design colors:
\colorlet{primary}{black!87!white}   % - high emphasis
\colorlet{secondary}{black!60!white} % - med emphasis
\colorlet{tertiary}{black!38!white}  % - disabled

\usepackage{titlesec}
\usepackage{titling}
\usepackage{fontspec}

\setmonofont{Roboto Mono}
\newfontfamily\titlefont[Color=primary, Scale=3.75]{Merriweather}
\setmainfont{Roboto}
\newfontfamily\sectionfont[Color=primary]{Roboto Condensed}
\newfontfamily\sectionfontlite[Color=primary]{Roboto Condensed-Light}
\newfontfamily\subsectionfont[Color=primary]{Roboto Slab}
\titleformat*{\section}{\LARGE\sectionfontlite}
\titleformat*{\subsection}{\large\subsectionfont}
\newcommand{\fakesubsection}[1]{\large{\subsectionfont{#1}}}
\titlespacing*{\section}{0pt}{6pt}{0pt}
\titlespacing*{\subsection}{0pt}{3pt}{0pt}

\newcommand{\tech}[1]{#1}

\newfontfamily{\annfont}[Color=secondary]{Roboto}
\newfontfamily\marginemph[Color=secondary]{Roboto Condensed}
\newcommand{\annotate}[1]{
	\marginpar{\small{\annfont#1}}
}

\newcommand{\icon}[1]{
	\raisebox{-3pt}{\def\svgwidth{12pt}\input{#1.pdf_tex}}
}

\usepackage{enumitem}
\newcommand{\listskills}[2]{
	\textbf{\marginemph #1}
	\begin{description}[topsep=0pt, noitemsep, labelsep=0pt]
	#2
	\end{description}
}

\usepackage{tabularx}

\usepackage{fontawesome}

\begin{document}
\thispagestyle{empty} % removes page number
\color{primary}

{\titlefont Matthew Barber}

\begin{tabularx}{\textwidth}{ @{} l c r @{} }
	\faGlobe{} Essex, UK & \faPhone{} 07951 415676 & \faEnvelope{} \href{mailto:quitesimplymatt@gmail.com}{quitesimplymatt@gmail.com}
\end{tabularx}

\section*{PROFILE}

\annotate{
	\textbf{\marginemph FIND ME ONLINE}
	\begin{description}[labelsep=1pt]
		\item \icon{github} \href{https://github.com/Honno/}{github.com/Honno}
		\item \icon{pypi} \href{https://pypi.org/user/Honno/}{pypi.org/user/Honno}
		\item \icon{kaggle} \href{https://www.kaggle.com/justhonno}{kaggle.com/justhonno}
		\item \icon{site} \href{https://matthewbarber.io/}{matthewbarber.io}
	\end{description}
}

I'm a BSc Computer Science graduate experienced in using \tech{Python} for data mining applications. I can efficiently explore, preprocess and model for data to identify the underlying patterns which lead to useful insights. I am able to engineer my own applications to accomplish bespoke tasks, from quick scripts to fully-fledged CLIs. I can work efficiently alone, being a disciplined and resourceful individual, always eager to improve my craft. With my employment and volunteering experiences necessitating cohesive teamwork, I enjoy working with and learning from my peers and can communicate my own ideas succinctly.

\section*{PROJECTS}

\annotate{
	\listskills{LANGUAGES}{
		\item Python
		\item SQL
		\item Lisp
		\item JavaScript
		\item Bash
		\item Java
	}
	
	\medskip
	
	\listskills{PACKAGES}{
		\item pandas
		\item NumPy
		\item SciPy
		\item scikit-learn
		\item Matplotlib
		\item Altair
		\item pytest
		\item tox
		\item Hypothesis
		\item PySpark
		\item Redis
		\item Jinja
		\item Click
	}
	
	\medskip
	
	\listskills{TOOLS}{
		\item Git
		\item Jupyter
		\item Weka
		\item Docker
		\item pre-commit
		\item GitHub Actions
		\item TravisCI
		\item AppVeyor
	}
}

\subsection*{\href{https://github.com/Honno/coinflip/}{coinflip}}

\tech{Python} library for assuring cryptographic randomness in RNGs. The implemented statistical tests use \tech{pandas} under the hood. A testing suite featuring \tech{pytest} and \tech{Hypothesis} ensures reliable results.

\subsection*{\href{https://github.com/Honno/epitope-classification}{Linear B-cell Epitope Classification}}

Essay on exploring, preprocessing and modelling for a dirty proteins dataset. I identified subtle duplication patterns in the data, resolved via a bespoke \tech{Python} script. \tech{Weka} was used to create Nearest Neighbour, Random Forests, Bayesian and Logistic Regression classifiers to find the most appropriate model for both equal and uneven cost scenarios.

\subsection*{\href{https://github.com/Joshgallagher/financial-analysis-stack}{Financial Analysis Stack}}

Creating regression models for stock market histories in \tech{Python}. Demonstrates how to build applications with distributed data via \tech{Hive} on \tech{HDFS}, and how to utilise parallel processing with \tech{PySpark}. All services are initialised as \tech{Docker} containers.

\subsection*{\href{https://github.com/Honno/events-site}{University Events Site}}

\tech{Node.js} site for students to manage university events. API and front-end routing was created in \tech{Express}, interacting with a \tech{MongoDB} data store.

\subsection*{\href{https://matthewbarber.io/gzip-quine/}{Recursive GZIP Bomb Tutorial}}

Comprehensive primer on the file format and compression algorithm theory involved in creating compressed file quines (i.e. extracts to an exact copy of itself, ad infinitum). Self-referential checksum was brute-forced by a multiprocessing \tech{Python} script. My file is used to smoke test Apache's Tika project, and exposed a MacOS vulnerability.

\section*{EXPERIENCE}

\fakesubsection{Ferndale Homeless Shelter} \hfill \textit{Nov 2015—Apr 2019}\\
\fakesubsection{Starbucks} \hfill  \textit{Sep 2018—Feb 2019}\\
\fakesubsection{OTS Homeless Shelter} \hfill \textit{Sep 2018—Feb 2019}\\
\fakesubsection{Adventure Island Theme Park} \hfill \textit{Jul 2017—Sep 2018}

\section*{EDUCATION}

1st (Honours) BSc Computer Science, Aston University

\end{document}